\title{Relational Query Languages}
\author{
        Artyom Desyatnikov
}
\date{\today}

\documentclass[12pt]{article}
\usepackage{amsmath}
\usepackage[utf8]{inputenc}
\usepackage{changepage}
\usepackage{caption}
\usepackage{float}

\captionsetup[table]{labelformat=empty}

\begin{document}
\maketitle

\tableofcontents
\newpage

\section{History}

In early 1960-x, with the advance of the first computers arose the need to
maintain the large dynamic collection of data. Conveniently, this demands
were met with the increasing availability of direct-access data storage devices.
From then on the database history begins. The first databases aimed to provide the
user with a means of programmatically store and traverse data organized in some
structure. Such was, for example, Information Management System developed in 1966 by IBM.
It operated an XML-like tree of nodes, each containing multiple homogeneous entries.
The API allowed to ``manually" navigate from a node to either its parent or
child and browse their contents on the way. That was why such databases were called navigational.
It was left for the user to organize their model in a manner to be efficiently traversable.
While this alone was said to be difficult in some cases, the evidence says it turned
even more complicated to maintain should the requirements
change to form new unexpected connections in the model.

The concept of relational database was introduced in 1970 by Edgar Codd in attempt to
describe an implementation-agnostic database model that would make for a query language
both declarative (in contrast to IMS' procedural approach) and adequately powerful.
In fact, there were 2 universal query languages introduced.
In the paper ``A Relational Model of Data for Large Shared Data Banks" (1970)
Codd presented \textbf{relational algebra}, a simple formal language to define steps of a query
execution. Later in 1970 Codd published another paper featuring \textbf{relational calculus},
a higher-level query language that essentially was the first order logic applied to
a database entities. Last but not the least, Codd proved that whatever can be expressed
in relational calculus, could as well be expressed in relational algebra and presented an
algorithm of such translation. Thus, relational calculus was intended as a user input
language while the translation algorithm was to convert the query into an algebraic form
suitable for execution. 

Codd pointedly criticized the existing DBMSs for the fact their queries and responses
lacked abstraction over the storage implementation details. What he, in turns, suggested
was to interpose an implementation-unaware query language, namely relational algebra,
between an application requirements and data storage specificities.
Given the flexibility, and therefore production value, Codd's approach proposed
it is a small wonder that his works were quickly picked up by the INGRES project
and soon by IBM themselves. Thus began the dominance of relational databases,
and it wasn't until mid 2000-s that it began to fade.

\section{Relational Model}

But the central notion of Codd's works was not the mere query language itself, but the
so-called relational database model. It is this model that Codd suggested as a replacement
for the network and hierarchal databases of 1960-x.

In order to approach the notion, first, let us revise the mathematical concept of relation.
The $k$-ary relation over $(D_1, D_2, ..., D_k)$ is defined as a subset of $D_1 \times D_2 \times ... \times D_k$.
This means that of all possible combinations of values from the corresponding domains
we select only those satisfying some condition, and call the set of them a relation.

The elements of the Cartesian product, taking the form of $(d_1,...,d_k)$, are called \textbf{tuples}.
The elements of tuples are called \textbf{attributes}.
Then, a $k$-ary relation can as well be said to be a set of tuples from $D_1 \times D_2 \times ... \times D_k$.

The relational database model is simply a set of named relations with a schema associated to each.
The schema is a list of attribute schemata, each specifying a name and a domain. The relation with the schema
$[(name : a_1,domain : D_1), ...(name : a_k,domain : D_k)]$
is naturally required to be defined on $D_1 \times D_2 \times ... \times D_k$.

For example, schema:

\begin{adjustwidth}{2cm}{}
		Author (AID integer, name string, age integer),\\
		Paper (PID string, title string, year integer),\\
		Write (AID integer, PID string)
\end{adjustwidth}

	and instance:

\begin{adjustwidth}{2cm}{}
		$\{<142, ``Knuth", 73>,<123, ``Ullman”, 67>, ...\}, $\\
		$\{<``181140pods”, ``Query containment”, 1998>, ...\},$\\ 
		$\{<123, ``181140pods”>,<142, ``193214algo”>, ...\}.$\\
\end{adjustwidth}

The notion of a relation comes very similar to what we are used to think of as a table.
A tricky difference one should be aware of, though, is that being technically a set, a relation
cannot contain a pair of duplicate tuples.

\section{Relational algebra}

Now let us turn our attention towards relational algebra, a language designed to query data
from relational model. The language essentially consists of several operations on relations
that yield in turns other relations. Thus, a query in relational algebra is a
possibly nested application of these predefined operations to either relations of a database
or constant tuple sets (like ${<1,``a">, <2,``b">}$) as the atomic values.
Naturally, a query result is also bound to be a relation.

\subsection{Basic operations}
The set of operations is redundant in a way, there being a subset of them sufficient to express
the rest. The list of them is as follows:
\begin{itemize}  
\item \textbf{Selection} $\sigma$

Selects the tuples from the relation matching some criteria. The criteria
is restricted to either equality or inequality between two attributes or between an attribute and a constant.
For example, $\sigma_{name=``Knuth"}(Author)$ will yield a relation containing a single tuple ${<142, ``Knuth”, 73>}$.
 
\item \textbf{Projection} $\pi$

Selects only the specified subset of attributes tuples from the relation matching some criteria. The criteria
is restricted to either equality or inequality between two attributes or between an attribute and a constant.
$\pi_{AID}(Write)$, for instance, will return IDs of all authors who have written a paper. Note, that given the
nature of set, there will be only a single occurrence of each AID, even though the author have written more than one paper.

\item \textbf{Rename} $\rho$

Renames the specified set of attributes. For example, $\rho_{AID \rightarrow AuthorID}$.
It is commonly put to use in order to make attribute name unique for the next operation.

\item \textbf{Cartesian product} $\times$

Yields all possible combination of tuples from two relations merged together. Note that it is required 
the attribute names be unique within a relation. To avoid ambiguity performing Cartesian product on two relations
sharing a common attribute name is prohibited. As an illustration, to have all possible authors older than
Knuth, one might first select the rows associated with Knuth: $$K=\sigma_{name=``Knuth"}(Author)$$
and then inspect the Cartesian product of the Knuth's age and all other authors
$$O=\rho_{age \rightarrow knuthsAge}(\pi_{age}(K)) \times Author $$
searching for age value more than the Knuth's:
$$\pi_{AID,name,age}(\sigma_{age > knuthsAge}(O)).$$

\item \textbf{Intersection and Union}

These two are straightforward analogs of their set-theoretic versions. Again, their application is restricted to
pairs of relations having the same attribute names, which in turns can be achieved by renaming them and dropping
off the extra ones with projection.

\end{itemize}


The listed operations along with all their possible compositions is what constitutes
the relational algebra.
More formally, an algebraic structure is defined as a carrier set and a list of operations on it.
Thus, the relational algebra of database D is an algebraic structure
consisting of the six aforementioned operations acting and
a closure of all relations from D under these operations serving as a carrier set.

\subsection{Additional operations}

There are several additional handy operations in relational algebra which are though perfectly
expressible through the six basic ones. They are given such credit for the fact at the database engine level
they can be implemented more efficiently than their translations and hence are to be preferred during
optimizations.

Take \textbf{natural join}, for example. It yields the set of all combinations of tuples of
the two given relations with their common attribute equal.

\begin{table}[H]
\caption{Employee}
\centering
\begin{tabular}{ l l l }
Name & EmpId & DeptName \\
\hline
Harry & 3415 & Finance \\
Sally & 2241 & Sales \\
George & 3401 & Finance \\
Harriet & 2202 & Sales \\
\end{tabular}
\end{table}

\begin{table}[H]
\caption{Dept}
\centering
\begin{tabular}{ l l }
DeptName & Manager \\
\hline
Finance & George \\
Sales & Harriet \\
Production & Charles \\
\end{tabular}
\end{table}

\begin{table}[H]
\caption{Employee $\bowtie$ Dept}
\centering
\begin{tabular}{ l l l l }
Name & EmpId & DeptName & Manager \\
\hline
Harry & 3415 & Finance & George \\ 
Sally & 2241 & Sales & Harriet \\
George & 3401 & Finance & George \\
Harriet & 2202 & Sales & Harriet \\
\end{tabular}
\end{table}

The natural join has its straightforward implementation as
$$\sigma_{A1=A2}(\rho_{A  \rightarrow  A1} R_1 \times \rho_{A \rightarrow A2} R_2)$$
However it would leave us having to traverse all the possible tuple combinations, which can and in most
cases must be avoided in a real DBMS. Instead of blindly searching for the matching tuples one
can index $R_1$ by the attribute $A$ to achieve $o(|R_1|\ max_{r_1 \in R_1} |\sigma_{A={r_1}.A} R_2 )|$
join complexity, essentially $o(|R_1|)$ in most cases, instead of $o(|R_1|\ |R_2|)$.

Probably the most challenging of the lot is the \textbf{division} operator. The result of $R_1 \div R_2$ with the
attribute sets $A_1$ and $A_2$ respectively is defined to have attributes $A_1 \setminus A_2$. It contains
the values of the attributes unique to $R_1$ which have all the combinations with entries from $R_2$
present in $R_1$. It can be thought of as a rought equivalent of a universal quantifier.
For the sake of illustration, here is how the division can be used to determine which students have
completed all the given tasks:

\begin{table}[H]
\caption{Completed}
\centering
\begin{tabular}{ l l }
Student & Task \\
\hline
Fred & Database1 \\
Fred & Database2 \\
Fred & Compiler1 \\
Eugene & Database1 \\
Eugene & Compiler1 \\
Sarah & Database1 \\
Sarah & Database2 \\
\end{tabular}
\end{table}

\begin{table}[H]
\caption{DBProject}
\centering
\begin{tabular}{ l }
Task \\
\hline
Database1 \\
Database2 \\
\end{tabular}
\end{table}

\begin{table}[H]
\caption{Completed $\div$ DBProject}
\centering
\begin{tabular}{ l }
Student \\
\hline
Fred \\
Sarah \\
\end{tabular}
\end{table}

Again, the division can be implemented in terms of the basic relational algebraic operations.
In this particular case it would be: $$\pi_{Student}(Completed) - \pi_{Student}(Completed -
(DBProject \times \pi_{Student}(Completed)))$$.

\section{Relational calculus}

Having discussed the basics of relational algebra, we should note that
creation of a low-level query execution planning language as it is
was not the final goal Codd had in mind. In his second paper he presented the readers with yet
another language for querying data from relational databases - the relational calculus.
The two languages had an almost completely unrelated syntax while bearing similar semantics.

By semantics of a query we mean a function from all possible databases to relations resulting
from the query execution on a given database. The semantics of the whole query language therefore
is the union of all its words' semantics. If two languages' semantics are equal sets such languages
are said to be equivalent in expressive power and if semantics of one is the superset of another's,
it is said to be more expressive. The proof of expressive equivalence of languages is usually given in form of
a bijection between them, or just an injection when expressive superiority is sought.

The aim of Codd's second work was to provide a human-readable language reducible to relational
algebra. Since he also did not cherish the idea of losing any of algebra's expressive power
by doing so, it was actually to be expressively equivalent to the relational algebra.
And indeed relational calculus, was proven to be such.

Now in fact there are two languages that the term ``relational calculus" might stand for. Codd gave his proof
regarding the so-called \textbf{tuple calculus}, while the second version known as \textbf{domain calculus}
was introduced 7 years later by two French computer scientists Michel Lacroix and Alain Pirotte.
The two languages undoubtedly have equivalent expressive power but the formal proof of the statement is
beyond our consideration.

The \textbf{tuple calculus} consists of queries in the form of
$$\{t_1.attr_1, ... t_k.attr_{i_k} | \phi(t_1, ..., t_k)\},$$
where $t_1, ..., t_n$ represent tuples
and $\phi$ is a second-order logic formula with relations of a database as unary predicates and
attribute equalities and inequalities $t_i.attr_1\ \rho\ t_j.attr_2$ as binary predicates.

Imagine the database from the division example being subjected to:
\[
\begin{aligned}
\{ compEntry.Student\ |\ Completed(compEntry) \wedge \\
	\forall proj DBProj(proj)  \rightarrow ( \exists comp \wedge \\ Completed(comp) \wedge \\
	comp.Task = proj.Task \wedge \\ comp.Student = compEntry.Student ) \}.
\end{aligned}
\]

Not exactly as clear as Codd probably expected, it is still obviously
equivalent to $Completed \div DBProject$.

The same semantics can be expressed in \textbf{domain calculus} as well, its query also being
a first-order formula, but with attributes as variables and $k$-ary relations instead of
unary ones.
\[
\begin{aligned}
\{ student | (\exists studentTask Completed(student, studentTask)) \wedge \\
	(\forall task DBProj(task)  \rightarrow  Completed(student, task)) \}.
\end{aligned}
\]

Note that some queries however are obviously inapplicable to the real-world databases, such as
$\{ p.PID | \neg Paper(p) \}$. These queries basically require the database to give all PIDs
in the world, and the only 'world' the database is aware of comes in form of the domains. An actually
correct response to the query stated would be a column, containing the union of all attribute domains of the
database except for $Paper$'s $PID$. Cases like this would be extremely difficult to implement and actually useless.
If nothing else, we would not want our database to output all $256^{2^{30}}$ strings it can represent. Thus,
to possess any interest to us, the queries are to yield results independent of attribute domains. At least,
so long as the domains hold all the values from relation - otherwise the database would be invalid.

To determine whether a given query is domain-dependent is an undecidable problem, i. e., there is no
algorithm that can tell whether a given relational calculus query is domain dependent \cite{kolatis}.
This is however unnecessary in pursuit our goal. Instead some simple-to-check restriction could be imposed
onto the tuple calculus queries in order to guarantee their domain independence. Since one is interested not
to lose any of its expressive power in the process, one would have to ensure, that
for every domain-independent query there is an equivalent one
conforming such restriction. Thus is the range-restricted queries, where each quantifier is followed
by some relation restricting its variable: $\forall x \in P$ (shorthand for
$\forall x (P(x)  \rightarrow  ...)$) or $\exists x \in P$ (shorthand for $\exists x (P(x) \wedge ...)$).
This way the only source of tuples to draw from would be strictly defined and domain independent.
One way to convert a query known to be domain-independent into a range-restricted form is to
specify the so-called active domain for each of its variables.
The active domain of a query is defined as a set of tuples from all possible relations together
with constants referenced in the query. If done so, the quantifier expressions
will take the form $ \forall(\exists) x \in \alpha(x) $, where
$\alpha(x)$ is a massive disjunction of $R(x)$ for all
relations $R$ in the database as well as one constant relation holding all possible constants from
the query for attributes of $x$ referenced in the query.
Then the disjunctions can be trivially separated so that the query fits the definition of range-restricted one.

\section{Codd's Theorem}

The main point of Codd's second paper was to prove range-restricted query language to be equivalent to relational algebra.
Besides, Codd's proof was constructive, that is to say it suggested an algorithm converting a
range-restricted query into that of relational algebra. The backwards conversion does not pose much of a problem
and will be left as an exercise to the reader.

As for the forward conversion, the formal algorithm starts
with first rewriting a query in either prenex normal form, i. e. with all its quantifiers placed in front.
The relation in question is initially set to simply be the Cartesian product of all relations of the query.
After that it is filtered according to the predicate-free part. Then the process starts that gradually shapes
the relation into the desired relational algebraic expression, by gradually applying one of the predefined formulas
to the quantifiers, right to left.

The list of the said formulas is usually regarded as tedious and therefore omitted. A thorough demonstration of
the conversion process, however, is given in \cite{cdate}.

\section{Limitations}

\newcommand{\Erdos}{\text{Erd{\H o}s}}

As powerful as the two languages might seem, there are certain limitations to their expressiveness. Consider
such metric as the \Erdos{} number, the numeric characteristic of a person that in a way represents their
``collaborative distance" to Paul \Erdos{}, one of the most productive authors of mathematical papers. The \Erdos{} number of
0 is assigned to Paul \Erdos{} himself. Those, who have at least once collaborated with him at writing a paper are said
to have the \Erdos{} number of 1. Their collaborators receive the number 2 unless they have already got 1 or 0. And so on.

Let us write a query for the Author-Paper-Write database from the previous examples, that selects IDs of authors with
the \Erdos{} number less or equal than 1. First, we should select the papers written by \Erdos{}:
\[A_0 = \pi_{AID}(\sigma_{name=``\Erdos{}"}Author)\]
\[P_1 = \pi_{PID}(A_0 \bowtie Write). \]

The desired set of author IDs will be:
\[ A_1 = \pi_{AID}(P_1 \bowtie Write). \]

Mechanically incrementing the indices we will be able to define a query
for an arbitrary large \Erdos{} number. Note, that while doing so the expression
will become increasingly nested but nevertheless perfectly executable.
However, in case one tries to construct a query selecting the authors
with the infinite \Erdos{} number, i.e. whose ``collaboration path" will never lead to Paul \Erdos{}, one is bound to fail
\cite{pichler}.

There is a whole classes of problems unsolvable by means of relational algebra and calculus, some of them dealing
with recursive structures or graph algorithms
likewise to the \Erdos{} number example, others related to grouping and aggregation of
tuples and some more.

\section{A note on SQL}

The SQL was intended as an end-user language featuring the relational model aspects, making for easy relational calculus
and algebra-like queries. Due to the lack of Codd's influence SQL turned out not to follow the concept of either of the 
two (even three) query languages discussed. Instead they have chosen to go for the mixture of all the approaches adding
some extra features on top of that, which lead to the language becoming extremely redundant.
Still, it was more than enough to implement all
relational operations in it which made SQL at least as expressively powerful as relational algebra.

Furthermore, such popular needs as that for aggregation or recursive queries have been tended to in SQL, thus
making the language superior to the relational algebra in terms of expressive power.

\begin{thebibliography}{9}

\bibitem{pichler}
  Reinhard Pichler,
  \emph{Datenbanktheorie}, Sommersemester 2016
	http://www.dbai.tuwien.ac.at/staff/pichler/dbt/

\bibitem{kolatis}
  Phokion G. Kolaitis University of California, Santa Cruz \& IBM Research-Almaden,
  \emph{Relational Databases, Logic, and Complexity}, 2009
	https://users.soe.ucsc.edu/$\sim$kolaitis/talks/gii09-final.pdf

\bibitem{cdate}
  C.J. Date,
  \emph{An Introduction to Database Systems}


\end{thebibliography}

\end{document}


